\chapter{Beispielkapitel}
\lipsum[1] \autocite{samulowitzBibliothekUndDokumentation2003,sarrafzadehKnowledgeManagementIts2010,saurWissenschaftlicheVerlageVersuch2016,scholzeOpenAccessUnd2016,schrettingerVersuchVollstaendigenLehrbuchs1829,seadleFragilityFutureLibrary2016,siegfriedNutzerbezogeneMarktforschungFuer2014}

\section{Unterkapitel (Struktur)}
\lipsum[3-4]

\subsection{Unter-Unterkapitel}
\lipsum[5-6]

\subsubsection{Unter-Unter-Unterkapitel}
\lipsum[66]

\paragraph{Paragraf}
\lipsum[66]

\subparagraph{Unterparagraf}
\lipsum[75]

\section{Unterkapitel (Mathematische Formel)}
\lipsum[1-5]
\begin{equation}
    \sin x = \sum\limits_{n = 1}^\infty  {\frac{{\left( { - 1} \right)^{n - 1} x^{2n - 1} }}{{\left( {2n - 1} \right)!}}}    
\end{equation}
\lipsum[1]