\chapter{Einführung}
\lipsum[5] \autocite{abbottProfessionalismFutureLibrarianship1998,alaviReviewKnowledgeManagement2001,altenhoenerZukunftFuerSaures2012,batesInformationProfessionsKnowledge2015,batesInformationScienceInvisible1999,begerRechtOeffentlichenZugaenglichmachung2016,bonteAusSachsenWelt2016,bucklandInformationSchoolsMonk2005,bucklandWhatKindScience2012,cuglianaComputationalTurnDigital2022,degkwitzBibliothekZukunftZukunft2016,degkwitzHaveDreamBibliothek2016,dietzeBibliotheksUndInformationswissenschaft1977}

\lipsum[6] 

\lipsum[75]

\section{Unterkapitel}
\lipsum[1-2]
\section{Unterkapitel (Tabellen und Grafiken)}
\lipsum[66]
\begin{table}[!htbp] %%Anmerkung: Diese Tabelle ist komplexer als für die meisten notwendig.
    \centering
    \caption{Ein Beispielstabelle ohne wirklich sinnvollen Inhalt, die aber ästhetisch anspruchsvoll und komplex dargestellt wird.}
    \begin{tabular}{lS[tight-spacing=true]S[scientific-notation=fixed, 
fixed-exponent=-4, round-precision=2, round-mode=places, table-format=1.2e-4,tight-spacing=true]SSS[scientific-notation=fixed, 
fixed-exponent=-4, round-precision=2, round-mode=places, table-format=1.2e-4,tight-spacing=true]S}\toprule
& \multicolumn{3}{c}{Kategorie 1} & \multicolumn{3}{c}{Kategorie 2}
\\\cmidrule(lr){2-4}\cmidrule(lr){5-7}
          & \multicolumn{1}{S}{P1}    & \multicolumn{1}{S}{P2}     & \multicolumn{1}{S}{P3}  & \multicolumn{1}{S}{P4}    & \multicolumn{1}{S}{P5}      & \multicolumn{1}{S}{P6}\\\midrule
Proband 1 & 110 & 1.21e-4 & 13.9 & 158 & 8.7e-5 & 5.6 \\
Proband 2 & 219 & 1.3e-5 & 16.2 & 315 & 1.42e-4 & 18.8 \\
\bottomrule
\end{tabular}
    \label{tab:beispiel}
\end{table}
Siehe \cref{tab:beispiel,fig:hu-logo}.

\noindent\lipsum[66]
\begin{figure}[!htbp]
    \centering
    \includesvg[width=40mm]{matter/titlepage/logo/hu_berlin_logo.svg}
    \caption{Das Logo der Humboldt-Universität zu Berlin, bestehend aus der Illustration der Gebrüder Humboldt mit dem umlaufendem Schriftzug \enquote{Humboldt-Universität zu Berlin}.}
    \label{fig:hu-logo}
\end{figure}

\noindent\lipsum[75]